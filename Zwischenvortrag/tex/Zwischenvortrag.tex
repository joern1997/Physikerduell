% !TeX program = lualatex
% Auch „german“ muss angegeben werden, damit die Sprache (z.B. in siunitx)
% auf Deutsch gestellt wird.
\documentclass[german, ngerman]{beamer}
\input{einstellungen.tex}

\title{Strukturbildung in Ökosystemen}
\subtitle{Zwischenvortrag: Anwendeung des Pseudospektralverfahrens \\ auf die Fischer-Gleichung}
% \subject wird von der Vorlage nicht direkt verwendet
%\subject{Thema}
% Autor angeben
\author{Johannes B. , Jörn S., Hauke H.,...}
% \institute wird von der Vorlage nicht direkt verwendet
\institute{}
\date{}
\keywords{Münster}

\begin{document}

% Titelfolie
\begin{frame}[plain]
	\titlepage
\end{frame}

\begin{frame}
\frametitle{Überblick}

\begin{itemize}
	\item Einführung: Nichtlinearität strukturbildender Prozesse
	\item Shortcut: Runge-Kutta und Räuber-Beute-Modell
	\item Galerkin-Methode
	\item FFT und deren Anwendung 
	\item Pseudospektralverfahren allgemein 
	\item (De)aliasing
	\item  Fischer-Gleichung: Interpretation und Lösung in 1D/2D
	
	
\end{itemize}
\end{frame}

\begin{frame}
\frametitle{Beispielbild}  
\begin{itemize}
	\item Beispiel
\end{itemize}
\begin{figure}
	\centering
	\includegraphics[scale=0.95]{Bilder/beispielbild.jpg}
\end{figure}


\end{frame}


\begin{frame}
\frametitle{Beispielgleichung}

\begin{block}  {Beispiel}
\begin{align*} 
G^{\mu \nu} = \frac{8\pi G}{c^4} T^{\mu \nu}                 
\end{align*}

\end{block}


\end{frame}


\begin{frame}
\frametitle{Quellen}

\begin{thebibliography}{9}


\bibitem{SRT}
Rainer Zufall (2012): \textit{Beispielquelle} (Beispielverlag)

\end{thebibliography}
\end{frame}

\end{document}
