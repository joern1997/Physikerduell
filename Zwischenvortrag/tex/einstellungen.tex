%%% Einstellungen zur richtigen Benutzung von wwustyle.sty
\usefonttheme{professionalfonts}

% Einstellungen der Schriftart (Meta Office Pro) für Text und Mathematik
% math**=sym angeben, damit auch diese Befehle die Schriftart Meta verwenden
\usepackage[mathrm=sym, mathit=sym, mathsf=sym, mathbf=sym]{unicode-math}
\setmainfont{MetaOffcPro}[Path=fonts/,
	Extension=.ttf,
	UprightFont=*-Norm,
	UprightFeatures={
		SmallCapsFont={MetaScOffcPro-Norm}
	},
	ItalicFont=*-NormIta,
	ItalicFeatures={
		SmallCapsFont={MetaScOffcPro-NormIta}
	},
	BoldFont=*-Bold,
	BoldFeatures={
		SmallCapsFont={MetaScOffcPro-Bold}
	},
	BoldItalicFont=*-Bold,
	BoldItalicFeatures={FakeSlant},
]
\setsansfont{MetaOffcPro}[Path=fonts/,
	Extension=.ttf,
	UprightFont=*-Norm,
	UprightFeatures={
		SmallCapsFont={MetaScOffcPro-Norm}
	},
	ItalicFont=*-NormIta,
	ItalicFeatures={
		SmallCapsFont={MetaScOffcPro-NormIta}
	},
	BoldFont=*-Bold,
	BoldFeatures={
		SmallCapsFont={MetaScOffcPro-Bold}
	},
	BoldItalicFont=*-Bold,
	BoldItalicFeatures={FakeSlant},
]

\setmathfont{Latin Modern Math}
% Meta Office Pro für die Bereiche nutzen, für die Glyphen existieren
\setmathfont{MetaOffcPro-Norm.ttf}[Path=fonts/,
range=up/{greek,Greek,latin,Latin,num}]
\setmathfont{MetaOffcPro-NormIta.ttf}[Path=fonts/,
range=it/{greek,Greek,latin,Latin,num}]
\setmathfont{MetaOffcPro-Bold.ttf}[Path=fonts/,
range=bfup/{greek,Greek,latin,Latin,num}]
\setmathfont{MetaOffcPro-Bold.ttf}[Path=fonts/, UprightFeatures={FakeSlant},
range=bfit/{greek,Greek,latin,Latin,num}]
% Symbole (leider enthält Meta Office Pro nicht das Symbol ∓)
\setmathfont{MetaOffcPro-Norm.ttf}[Path=fonts/,
range={`\+, `\-, `\×, `\÷, `\⋅, `\*, `\/, `\⁄, `\±,
	`\=, `\≠, `\≈, `\<, `\>, `\≤, `\≥, \partial, `\∞, `\†, `\‡,
	`\%, `\‰, `\!, `\?, `\., `\,, `\:, `\;, `\&, `\#, `\@,
	`\§, `\€, `\$, `\£, `\¥, `\©, `\®,},
]

% Spracheinstellung
\usepackage{polyglossia}
\setmainlanguage{german}

% Offizielles WWU-LaTeX-Paket für Präsentation (leicht modifiziert)
% Mögliche Optionen:
% - english: Verwendet englischen Claim („living.knowledge“)
% - Verschiedene Farbvarianten:
%   pantoneblack7, pantone312, pantone315, pantone3282, pantone369, pantone390,
%   pantone396, pantoneprozessyellow
% - Verschiedene Titel-Motive:
%   - belltower (Standard-Wert): Glockenturm des Schlosses
%   - wedge: Textkeil
%   - prinz: WWU-Schriftzug auf dem Prinzipalmarkt (Foto)
% - inverse: Inverses Titelbild (Weiß auf farbigem Hintergrund statt Farbe auf
%            weißem Hintergrund)
% - wide: Seitenverhältnis 16:10 verwenden (statt Standardwert 4:3)
\usepackage[pantone312]{wwustyle-mod}
% Typographische Verbesserungen (Verbot mancher Ligaturen)
\usepackage{selnolig}
% Typographische Verbesserungen (Mikrotypographie)
\usepackage{microtype}

% Daten/Zeiten formatieren
\usepackage[useregional]{datetime2}
% „Schöne“ Brüche im Fließtext mit \sfrac
\usepackage{xfrac}
% Ermöglicht die Nutzung von „\SI{Zahl}{Einheit}“
\usepackage{siunitx}
% Automatisches Umwandeln von Anführungszeichen
\usepackage{csquotes}

% Farben ermöglichen
\usepackage{xcolor}
% Paket für Bilder-Einbindung (EPS, PNG, JPG, PDF)
\usepackage{graphicx}
% .tex-Dateien mit \includegraphics einbinden
\usepackage{gincltex}
% Bessere Verarbeitung von Dateinamen für \includegraphics etc.
\usepackage{grffile}


% latex
\renewcommand{\arraystretch}{1.3}
% graphicx
% Standardmäßig „keepaspectratio“ verwenden
% s. https://tex.stackexchange.com/a/91619/51235
\setkeys{Gin}{keepaspectratio}
% hyperref
\hypersetup{unicode}
% siunitx
\sisetup{
	locale=DE,
	binary-units,
	quotient-mode=fraction,
	per-mode=fraction,
	fraction-function=\sfrac,
	detect-weight
}
% csquotes
\MakeOuterQuote{"}


%%%%%%%%%%%%%%%%%%%%%%%%%%%%%%%%%%%%%%%

% Zusätzliche Einstellungen/Befehle
\let\strong\textbf
\newcommand{\email}[1]{\href{mailto:#1}{\texttt{#1}}}
\newfontfamily\DejaSans{DejaVu Sans}
